% arara: lualatex: { shell: yes }
% arara: lualatex: { shell: yes }
\documentclass[
	spanish,
	9pt,
	xcolor={table,svgnames},
	handout,
	aspectratio=1610,
	% ignorenonframetext
]{beamer}

\usepackage[spanish]{babel}
\usepackage[audience=
	%english
	spanish
	%german
]{beameraudience}

\usepackage[
	cursointroductorio,
	fullpagenumbering
]{dunestyle-beamer}
\usepackage{booktabs}
\usepackage{longtable}
\usepackage[fixed]{fontawesome5}

\usepackage[backend=biber,style=numeric, defernumbers=true, sorting=ynt,maxbibnames=4,maxcitenames=4]{biblatex}
% \addbibresource{dune-doxygen-references.bib}

\title{
	Introducción a~\git{}
}
\subtitle{
	Una sistema de control de versiones descentralizado
}

\author{}
\institute[]
{
	\noindent
	Practique los ejemplos en Gitpod

	\href{https://gitpod.io\#/https://github.com/cpp-review-dune/dune-basics}{\includegraphics[width=4.5cm]{open-in-gitpod}}

	\vfill

	\begin{minipage}{0.12\paperwidth}
		Disponible en
	\end{minipage}%
	\hfill%
	\begin{minipage}{0.18\paperwidth}
		\href{https://github.com/cpp-review-dune}{\includegraphics[width=1.2cm]{GitHub-Mark-120px-plus}}
	\end{minipage}

	\vfill

	\begin{minipage}{0.26\paperwidth}
		¡Únete al grupo en Telegram!
	\end{minipage}%
	\hfill%
	\begin{minipage}{0.04\paperwidth}
		\href{https://t.me/joinchat/OsfYP1xnFlxjN2Ix}{\includegraphics[width=1.2cm]{telegram-logo}}
	\end{minipage}

}

\setbeamertemplate{bibliography item}{%
	\ifboolexpr{ test {\ifentrytype{book}} or test {\ifentrytype{mvbook}}
		or test {\ifentrytype{collection}} or test {\ifentrytype{mvcollection}}
		or test {\ifentrytype{reference}} or test {\ifentrytype{mvreference}} }
	{\setbeamertemplate{bibliography item}[book]}
	{\ifentrytype{online}
		{\setbeamertemplate{bibliography item}[online]}
		{\setbeamertemplate{bibliography item}[article]}}%
	\usebeamertemplate{bibliography item}}

\defbibenvironment{bibliography}
{\list{}
	{\settowidth{\labelwidth}{\usebeamertemplate{bibliography item}}%
		\setlength{\leftmargin}{\labelwidth}%
		\setlength{\labelsep}{\biblabelsep}%
		\addtolength{\leftmargin}{\labelsep}%
		\setlength{\itemsep}{\bibitemsep}%
		\setlength{\parsep}{\bibparsep}}}
{\endlist}
{\item}

\newcommand{\git}{%
	\begingroup\normalfont
	\includegraphics[height=2\fontcharht\font`B]{git}%
	\endgroup
}

\newcommand{\diode}{%
	\begingroup\normalfont
	\includegraphics[height=2\fontcharht\font`B]{diode}%
	\endgroup
}

\newcommand{\MVAt}{{\usefont{U}{mvs}{m}{n}\symbol{`@}}}

\begin{document}

\frame[plain,noframenumbering]{\titlepage}

\begin{frame}
	\frametitle{Vídeos recomendados en \texttt{diode.zone}~\diode{}}
	\begin{itemize}
		\item[\diode{}] \href{https://diode.zone/w/fd7db693-38be-46fd-871d-dfb545953231}{Introducción, clonar un repositorio en Windows 10 con Visual Studio Community 2019}.
		\item[\diode{}] \href{https://diode.zone/w/6c6997e0-73d9-4346-b9d3-6643edf5942a}{Clonar un repositorio, añadir cambios, hacer confirmación y subir objetos a GitHub}.
		\item[\diode{}] \href{https://diode.zone/w/7nbonxgvEr7FMTgkKrEu4N}{¿Qué es un control de versiones?}
	\end{itemize}
\end{frame}

\begin{frame}[fragile]
	\frametitle{Instalación de~\git{}}
	En la distribución Arch Linux, el binario se instala con la siguiente directiva
	\begin{verbatim}
[user@host somedir]$ sudo pacman -Syu git
    \end{verbatim}
	Si ya tiene instalado \lstinline|dune-common|, entonces ya tendrá instalado \lstinline|git| como dependencia.

	\

	En macOS instale \href{https://brew.sh}{\lstinline|homebrew|} y a continuación escriba en el emulador de terminal \lstinline|brew install git|. Para Windows 10/11, siga este \href{https://courses-2020-1.gitlab.io/abet-project/guideofuser.pdf}{\alert{tutorial}}.
\end{frame}

\begin{frame}
	\frametitle{¿Qué características tiene un control de versiones?}

	Según la entrada de~\href{https://es.wikipedia.org}{Wikipedia}, un sistema de control de versiones debe proporcionar:

	\begin{quote}\itshape
		\begin{itemize}
			\item Mecanismo de almacenamiento de los elementos que deba gestionar.
			\item Posibilidad de realizar cambios sobre los elementos almacenados.
			\item Registro histórico de las acciones realizadas con cada elemento o conjunto de elementos (normalmente pudiendo volver o extraer un estado anterior del producto).
		\end{itemize}
		\
		\noindent\hfill\color{DarkBlue}$\left(\faIcon{wikipedia-w}\right)$
	\end{quote}

\end{frame}

\begin{frame}[fragile]
	\frametitle{Autentificación}
	Desde el 13 de agosto del 2021\footnote{https://github.blog/2020-12-15-token-authentication-requirements-for-git-operations}, GitHub no aceptará la autentificación vía contraseña, por lo que podemos continuar vía token o claves SSH. Optaremos por la última opción\footnote{https://docs.github.com/es/github/authenticating-to-github/connecting-to-github-with-ssh}.

	En la distribución Arch Linux, el binario se instala con la siguiente directiva
	\begin{verbatim}
[user@host somedir]$ sudo pacman -Syu openssh
    \end{verbatim}
	Es un \href{https://www.openssh.com}{port del software SSH} (Secure Shell) del proyecto OpenBSD bajo la licencia BSD.
\end{frame}

\begin{frame}[fragile]
	\frametitle{}
	\begin{verbatim}
		[user@host somedir]$ git config --list
	\end{verbatim}
	En el directorio \lstinline|~/.ssh| tendrá \lstinline|id_ed25519.pub| \lstinline|id_ed25519| \lstinline|config| \lstinline|known_hosts|.
	\lstinline|man ssh|.
\end{frame}
%https://git-scm.com/book/es/v2/Git-en-el-Servidor-Generando-tu-clave-p%C3%BAblica-SSH

\begin{frame}[fragile]
	\frametitle{Utilizando el sistema de firma de clave pública \href{https://ed25519.cr.yp.to}{Ed25519}}
	Abra un emulador de terminal y escriba
	\begin{verbatim}
		[user@host somedir]$ ssh-keygen -t ed25519 -C "<comentario>"
	\end{verbatim}
	También podría optar el cifrado RSA, de la siguiente manera
	\begin{verbatim}
		[user@host somedir]$ ssh-keygen -t rsa -b 4096 -C "<comentario>"
	\end{verbatim}
	En caso de que no desee fijar una contraseña continue con la tecla Intro hasta acabar las preguntas.

	Si luego desea que no le pida la contraseña en cada intento, agregue en el archivo \lstinline|~/.gitconfig| lo siguiente:
	\begin{verbatim}
[credential]
        helper = cache --timeout=3600
	\end{verbatim}

	Como paso final, copie su clave pública (.pub) en la siguiente caja
\end{frame}

\begin{frame}
	\frametitle{¿Por qué usar un control de versiones?}
	Si no tiene un SCV, las cosas se salen de control muy
	rápidamente.

	Las razones adecuadas son:
	\begin{description}
		\item[Colaboración] el SCV permite que un equipo de personas
			trabaje en el mismo proyecto al mismo tiempo.
		\item[Almacenamiento de versiones] el SCV gestiona todas las
			versiones de todos los archivos, los almacena, los nombra y
			puede recuperarlos.
		\item[Seguimiento de los cambios] el SCV registra con precisión
			lo que se cambió y se debe dar una razón para el cambio.
		\item[Restaurar cambios y rutas de regresión] el SCV permite
			restaurar archivos individuales o grupos de archivos a una
			versión anterior.
	\end{description}
\end{frame}

\begin{frame}
	\frametitle{Terminología
		\fontspec[Renderer=Harfbuzz]{NotoColorEmoji.ttf}📖}
	\begin{description}
		\item[Repositorio]

			Es una base de datos que contiene toda la información acerca de las versiones de los archivos en un proyecto, ubicado en una carpeta oculta \lstinline|.git|.

		\item[Confirmación]

			Instantánea global de todos los archivos del proyecto con descripción de los cambios con respecto a la versión anterior.

		\item[Rama]

			Secuencia de commits que describe una rama (línea) de desarrollo. Un repositorio puede contener más de uno. La rama estándar se llama \lstinline|main| o \lstinline|master|.

		\item[Tag]

			Nombre persistente (identificador) para un commit, por ejemplo, cuando hace un lanzamiento público.
	\end{description}
\end{frame}

\begin{frame}[fragile]
	Es una buena idea presentarse a Git con su nombre y dirección de
	correo electrónico pública antes de realizar cualquier operación.
	La forma más sencilla de hacerlo es:

	\begin{verbatim}
[user@host somedir]$ git config --global user.name <user>
[user@host somedir]$ git config --global user.email <user@example.com>
		\end{verbatim}
	Remplace \lstinline|<user>| por el
	usuario de GitHub (sin \MVAt) y
	\lstinline|<user@example.com>| por el correo
	registrado en GitHub.
\end{frame}


\begin{frame}[fragile]
	\frametitle{Confirmaciones
		\fontspec[Renderer=Harfbuzz]{NotoColorEmoji.ttf}🔖}

	Una confirmación contiene
	\begin{itemize}
		\item una instantánea de todos los archivos.
		\item la fecha de creación.
		\item el nombre y contenido con respecto al autor de los cambios.
		\item una lista de cambios.
		\item una lista de referencias a confirmaciones principales.
	\end{itemize}

	Las confirmaciones son identificadas usando un hash
	(\href{https://es.wikipedia.org/wiki/Suma_de_verificaci%C3%B3n}{suma de verificación}) de su contenido, por ejemplo
	\lstinline|5cd006a1c044b786ea8196636cd0b62e296dbbe5|.

	Los hash de confirmación pueden abreviarse siempre que la abreviatura
	sea única.

	Las confirmaciones no pueden cambiar: contenido diferente
	$\Rightarrow$ valor hash diferente.

\end{frame}

\begin{frame}
	\frametitle{Ramas \fontspec[Renderer=Harfbuzz]{NotoColorEmoji.ttf}🌲}

	\begin{itemize}
		\item Una rama es una línea de desarrollo dentro de un
		      repositorio.
		\item Los repositorios pueden contener un número arbitrario de
		      ramas.
		\item \lstinline|git status| puede decirle en qué rama
		      se encuentra.
		\item Una rama siempre apunta a un \lstinline|commit|.
		\item La creación de una nueva confirmación guarda la
		      confirmación actual como principal y la rama luego apunta
		      la nueva confirmación.
	\end{itemize}

\end{frame}

\begin{frame}
	\frametitle{Varios repositorios
		\fontspec[Renderer=Harfbuzz]{NotoColorEmoji.ttf}📡}

	\begin{itemize}
		\item \faIcon{git} puede sincronizar cambios entre repositorios.
		\item Los repositorios adicionales se llaman remotos.
		\item Las ramas de un repositorio remoto tienen el nombre del
		      repositorio como prefijo.
		\item Puede clonar un repositorio remoto para obtener su contenido.
		\item No se pueden ver ramas de repositorios remotos directamente.
		\item En la primera revisión, git crea una rama de seguimiento.
		\item Los nuevos cambios se pueden descargar y combinar usando
		      \lstinline|git pull|.
	\end{itemize}
\end{frame}

% \begin{frame}\transblindsvertical
% 	\frametitle{Referencias}
% 	%------------------------------------------------------------ 1
% 	\only<1>{
% 		\begin{itemize}
% 			\item Libros
% 			      \nocite{*}
% 			      \printbibliography[heading=none,keyword=book]
% 		\end{itemize}
% 	}
% 	%------------------------------------------------------------ 2
% 	\only<2>{
% 		\begin{itemize}
% 			\item Artículos
% 			      \printbibliography[heading=none,keyword=paper]
% 		\end{itemize}
% 	}
% 	%------------------------------------------------------------ 3
% 	\only<3>{
% 		\begin{itemize}
% 			\item Sitios web
% 			      \printbibliography[heading=none,keyword=online]
% 		\end{itemize}
% 	}
% \end{frame}

\end{document}