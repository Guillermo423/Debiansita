\input{dune-gmsh-preamble}

\begin{document}

\frame[plain,noframenumbering]{\titlepage}

\begin{frame}[fragile]
	\frametitle{Acerca de \texttt{gmsh}}
	Es un software de código abierto para mallas tridimensionales de elementos
	finitos y visualización de datos, cuenta su propio lenguaje de scripting y cuatro módulos:

	\begin{description}
		\item[\texttt{geometry}]

			capa geométrica orientado a objetos para la creación de formas tridimensionales. Motores disponibles: Gmsh CAD y OpenCascade.

		\item[\texttt{meshing}]

			un módulo de mallado

		\item[\texttt{solving}]

			un módulo solucionador de ecuaciones

		\item[\texttt{post-processing}]

			un módulo de postprocesador.
	\end{description}

	\

	Estamos interesados en usar la interfaz de línea de comandos
	\verb|gmsh|, además de conocer dicho lenguaje de scripting así como
	sus APIs.

	\

	Un \verb|archivo.geo| $\stackrel{\texttt{gmsh}}{\longrightarrow}$ \verb|archivo.msh|.
	El primero contiene información de la geometría y parámetros de la malla.
\end{frame}

\begin{frame}[fragile]
	Si estamos usando el repositorio \verb|arch4edu| o el \verb|aur|, instale con \verb|yay -Sy gmsh gmsh-docs|.\scriptsize
	\begin{verbatim}
		Repository      : arch4edu
		Name            : gmsh
		Version         : 4.8.4-1
		Description     : An automatic 3D finite element mesh generator with pre and post-processing facilities.
		Architecture    : x86_64
		URL             : http://gmsh.info/
		Licenses        : custom
		Groups          : None
		Provides        : None
		Depends On      : fltk  med  opencascade  cairo  metis  alglib  ann  glu
		Optional Deps   : gmsh-docs: docs for gmsh
		python2: for gmsh.py
		python: for gmsh.py
		julia: for gmsh.jl
		Conflicts With  : None
		Replaces        : None
		Download Size   : 7.80 MiB
		Installed Size  : 38.88 MiB
		Packager        : calil (on behalf of Jingbei Li) <i@jingbei.li>
		Build Date      : Sun 02 May 2021 02:12:25 PM -05
		Validated By    : MD5 Sum  SHA-256 Sum  Signature
	\end{verbatim}
\end{frame}

\begin{frame}[fragile]
	\frametitle{Comandos elementales}
\end{frame}

\begin{frame}
	\frametitle{Objetivos de esta introdución}
	A number of advanced.
	\begin{itemize}
		\item Correr los \href{https://github.com/orgs/cpp-review-dune/packages}{imágenes prediseñadas} con el programa \lstinline{docker} en sistemas tipo Unix.
		\item Crear una malla con \lstinline{gmsh} a través de la API (C/C++, Python, Julia).
		\item Realizar los ejemplos resueltos del \textsc{Dune book}.
	\end{itemize}
\end{frame}

\begin{frame}
	\frametitle{Conociendo el sistema operativo \href{https://archlinux.org}{Arch Linux}}

	Es una distribución GNU/Linux de propósito general, desarrollada\footnote{El líder del proyecto es \href{https://wiki.archlinux.org/title/User:Anthraxx}{\texttt{Anthraxx}}, desarrollador alemán del kernel \href{https://github.com/anthraxx/linux-hardened}{\texttt{linux-hardened}}.} independientemente para procesadores x86-64, que se adhiere a los principios de simplicidad, modernidad, pragmatismo, centrado en usuarios y versatilidad.

	\

	Proporciona las últimas versiones estables de la mayoría del software siguiendo el modelo de \href{https://en.wikipedia.org/wiki/Rolling\_release}{lanzamiento continuo}, no existen versiones como en Ubuntu 20.04, 21.10, 22.04, etc.

	\

	Arch está respaldado por \lstinline{pacman}, un gestor de paquetes ligero, sencillo y rápido, que permite actualizar todo el sistema con una orden.
	Los scripts \href{https://wiki.archlinux.org/title/PKGBUILD}{PKGBUILD} aportados por la comunidad para la elaboración desde las fuentes, como los módulos de DUNE, se encuentran en el \href{http://aur.archlinux.org}{\emph{Arch User Repository}}.

	\begin{table}[ht!]
		\caption{Comparación de la línea de comando del gestión de software (fuente: \url{wiki.archlinux.org})}
		\centering\footnotesize
		\begin{tabular}{cccp{50pt}cc}
			\toprule
			Acción             & Arch                    & Red Hat/Fedora          & Debian/Ubuntu                                          & SLES/openSUSE           & Gentoo
			\tabularnewline
			\midrule
			Instala paquetes   & \lstinline|pacman -S|  & \lstinline|dnf install|  & \lstinline|apt install|                                 & \lstinline|zypper install|  & \lstinline|emerge -a|
			\tabularnewline
			Elimina paquetes   & \lstinline|pacman -Rs|  & \lstinline|dnf remove|  & \lstinline|apt remove|                                 & \lstinline|zypper remove|  & \lstinline|emerge -C|
			\tabularnewline
			Busca paquetes     & \lstinline|pacman -Ss| & \lstinline|dnf search| & \lstinline|apt search|                                & \lstinline|zypper search| & \lstinline|emerge -S|
			\tabularnewline
			Actualiza paquetes & \lstinline|pacman -Syu| & \lstinline|dnf upgrade| & \lstinline|apt update \&\&|\newline\lstinline|apt upgrade| & \lstinline|zypper update| & \lstinline|emerge -u world|
			\tabularnewline
			\bottomrule
		\end{tabular}
	\end{table}

\end{frame}

\begin{frame}

	En esta ocasión hemos elegido Arch Linux como ambiente de trabajo porque tiene disponible una gran variedad de módulos de DUNE.
	Es recomendable habilitar el repositorio \href{https://wiki.archlinux.org/title/Unofficial\_user\_repositories\#arch4edu}{\texttt{arch4edu}}\footnote{Administrado por Jingbei Li de la Universidad de Tsinghua.}.
\end{frame}

\begin{frame}[fragile]
	\begin{lstlisting}
    gitpod ~/dune-basics $ cd
  \end{lstlisting}
\end{frame}

\begin{frame}[fragile]
	\begin{figure}
		\centering
		\buildVoronoiBW[tikz,delaunay=show,styleDelaunay=dashed]
		{(0.3,0.3);(1.5,1);(4,0);(4.5,2.5);(1.81,2.14);(2.5,0.5);(2.8,1.5);(0.1,2);(1.5,-0.3)}
	\end{figure}
\end{frame}

%\lstinputpath{../../../sandbox}

\begin{frame}[fragile]
	\frametitle{Classes}

	\lstinputlisting[
		caption={Programa \texttt{hello-linux.cc}.},
		label=hello-linux.cc,
	]{../../../sandbox/hello-linux.cc}


\end{frame}

\begin{frame}\transblindsvertical
	\frametitle{Referencias}
	%------------------------------------------------------------ 1
	\only<1>{
		\begin{itemize}
			\item Libros
			      \nocite{*}
			      \printbibliography[heading=none,keyword=book]
		\end{itemize}
	}
	%------------------------------------------------------------ 2
	\only<2>{
		\begin{itemize}
			\item Artículos
			      \printbibliography[heading=none,keyword=paper]
		\end{itemize}
	}
	%------------------------------------------------------------ 3
	\only<3>{
		\begin{itemize}
			\item Sitios web
			      \printbibliography[heading=none,keyword=online]
		\end{itemize}
	}
\end{frame}

\end{document}

% https://albertsk.files.wordpress.com/2012/12/gmsh_tutorial.pdf
% https://indico.cern.ch/event/591096/contributions/2388462/attachments/1390640/2118262/main.pdf
% http://jsdokken.com/converted_files/tutorial_gmsh.html
% https://github.com/alejohg/tutoriales-gmsh
% https://github.com/stvanegasgi/Tutorial-Gmsh-4.6.0
% https://github.com/ahojukka5/gmshparser
% https://victorsndvg.github.io/FEconv/formats/gmshmsh.xhtml
% https://gmsh.info/doc/texinfo/gmsh.html
% https://github.com/sunnyerteit/inventasFlow
% https://ctan.dcc.uchile.cl/macros/luatex/latex/luamesh/doc/luamesh-doc.pdf
% https://gmsh.info/doc/course/general_overview.pdf
% https://indico.cern.ch/event/591096/#10-a-getdp-intro
% http://freshkiss3d.gforge.inria.fr/index.html
% https://framabook.org/docs/Code_Aster/beginning_with_code_aster-jp_aubry-20190129.pdf
% https://www.youtube.com/watch?v=IuMa4bQI1Fg
% https://www.youtube.com/playlist?list=PLbiOzt50Bx-l2QyX5ZBv9pgDtIei-CYs_
% https://www.youtube.com/watch?v=oJB7ODwBoQ8
% https://www.youtube.com/watch?v=LsL39CuPFx0
% https://www.youtube.com/watch?v=Lhgw4dSEJbQ
% https://www.youtube.com/watch?v=9w4Te298zsc
% https://www.youtube.com/watch?v=R1S6cAwasuM
% https://www.youtube.com/playlist?list=PLxT-itJ3HGuVHvjEqKVwLrcllRd5P4pXm
% http://gmsh.info/dev/doc/texinfo/gmsh.txt