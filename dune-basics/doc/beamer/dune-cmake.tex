\input{dune-cmake-preamble}

\begin{document}

\frame[plain,noframenumbering]{\titlepage}


\begin{frame}[fragile]
	El fichero \lstinline{CMakeLists.txt} nos indica el mínimo de versión de cmake que soporta esta configuración.
	En este script definiremos todo acerca de la construcción, desde el código fuente y objetivos, pasando por las pruebas, empaquetamiento, etc.
	Este script empieza con
	\href{https://cmake.org/cmake/help/latest/command/cmake_minimum_required.html}{\lstinline{cmake_minimum_required}}
	\href{https://cmake.org/cmake/help/latest/command/project.html}{\lstinline{project}}
	\href{https://cmake.org/cmake/help/latest/command/set.html}{\lstinline{set}}
	\href{https://cmake.org/cmake/help/latest/command/add_executable.html}{\lstinline{add_executable}}
\end{frame}

\begin{frame}[fragile]

\inputminted{cmake}{CMakeList.txt.sample}

\end{frame}

\begin{frame}
	\frametitle{El comando \lstinline{duneproject}}
	% https://gitlab.dune-project.org/core/dune-common/-/raw/master/bin/duneproject
	Es un asistente en el lenguaje \lstinline{bash} que se encuentra en \lstinline{/usr/bin/duneproject}\ldots

\end{frame}

\begin{frame}\transblindsvertical
	\frametitle{Referencias}
	%------------------------------------------------------------ 1
	\only<1>{
		\begin{itemize}
			\item Libros
			      \nocite{*}
			      \printbibliography[heading=none,keyword=book]
		\end{itemize}
	}
	%------------------------------------------------------------ 2
	\only<2>{
		\begin{itemize}
			\item Artículos
			      \printbibliography[heading=none,keyword=paper]
		\end{itemize}
	}
	%------------------------------------------------------------ 3
	\only<3>{
		\begin{itemize}
			\item Sitios web
			      \printbibliography[heading=none,keyword=online]
		\end{itemize}
	}
\end{frame}

\end{document}
%https://cliutils.gitlab.io/modern-cmake/
%https://www.dune-project.org/sphinx/content/sphinx/core-2.7/