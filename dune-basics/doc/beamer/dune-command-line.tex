\input{dune-command-line-preamble}

\begin{document}

\frame[plain,noframenumbering]{\titlepage}

\begin{frame}
	\frametitle{A~\gnu{}A~\linux{}.}
	GNU/Linux es un sistema operativo de código abierto. Este fue diseñado para ser similar a UNIX.
	UNIX
	Linux se refiere al kernel.
	Linux realmente no es un UNIX.
\end{frame}

\begin{frame}

	\frametitle{Historia del desarrollo de los sistemas Unix}

	\begin{itemize}
		\item
		      La primera versión de Unix fue escrito por Dennis Ritchie y Ken Thompson en $1969$.
		\item
		      Durante los $1970$s, los laboratorios Bell no permitían vender Unix comercialmente, sin embargo, lo distribuyeron a universidades a bajo costo.
		\item
		      La Universidad de California en Berkeley añadió sus propias implementaciones, el resultado es conocido como la versión BSD de Unix.
		\item
		      Tempranamente en $1980$s, AT\&T firmó un acuerdo en el que permitía vender System III, la primera versión Unix de AT\&T.
		\item
		      Su popularidad se incrementó en esta década, varias compañías modificaron el código de Unix para crear sus propias variantes. Ejemplos son SunOS, Ultrix, HP-UX.
		\item
		      Estas variantes fueron similares, pero a su vez existían ligeras diferencias en el conjunto de cabeceras y listas diferentes de funciones en las bibliotecas del sistema.
		\item
		      El estándar POSIX ayudó a eliminar estas diferencias, sin embargo, POSIX trajo nuevas características en diferentes tiempos, conllevando a nuevas variantes.
		\item Los autores querían que sus programas corran en una amplia variedad de Unix, la idea general fue usar \texttt{\#ifdef}, sin embargo, incrementaba la dificultad de conocer qué versiones tenían ciertas características.
	\end{itemize}
\end{frame}

\begin{frame}
	\frametitle{Empezando con GNU/Linux}
	Nos otorga un entorno de programación con un programa base llamado shell que nos permite comunicarnos entre el kernel y las aplicaciones de usuario, por lo que controlamos el sistema operativo.
	Tiene un lenguaje de scripting que nos permite automatizar tareas que sigue el estándar POSIX.

	\begin{itemize}
		\item Abrir terminal\quad\keys{\ctrl+\Alt+t}.
		\item Obtener los comandos digitados recientemente\quad\keys{\arrowkeyup}/\keys{\arrowkeydown}.
		\item Limpiar la consola \quad\keys{\ctrl+l}.
		\item Moverse entre palabras \quad\keys{\Alt+f}/\keys{\Alt+b}.
		\item Moverse al inicio/final del comando \quad\keys{Home}/\keys{End}.
		\item Buscar comandos utilizados anteriormente\quad\keys{\ctrl+r}/\keys{\ctrl+s}.
		\item Interrumpir la ejecución de un programa\quad\keys{\ctrl+c}.
		\item Enviar a segundo plano un proceso\quad\keys{\ctrl+z}.
		\item Autocompletar el nombre de un archivo\quad\keys{\tab}.
		\item Despliega el número de posibilidades para el comando\quad\keys{\tab+\tab}.
	\end{itemize}
\end{frame}

% Open Source

\begin{frame}
	\begin{itemize}
		\item
		      Linux comenzó como un proyecto de pasatiempo por Linus Torvalds en $1991$.
		\item
		      La fuente se hizo disponible libremente y otros se unieron para formar este sistema operativo de vanguardia.
		\item
		      Como fue creado desde cero, los primeros usuarios pudieron influir en el rumbo del proyecto y asegurarse de no cometer los mismos errores de otros UNIXes.
		\item
		      Los proyectos de software, como \href{https://dune-project.org}{Dune Project}, toman la forma de código fuente.
		\item
		Es un conjunto de instrucciones de cómputo legibles para el humano.
		\item Linux está siendo escrito principalmente en C.
	\end{itemize}
\end{frame}

\begin{frame}[fragile]
	\frametitle{Comandos básicos}

	Existen algunas convenciones acerca del directorio de trabajo, tiene la forma \lstinline|/foo/bar|.
	\begin{verbatim}
    [user@host somedir]$ pwd
    /home/user/somedir
    \end{verbatim}
	Cada directorio tiene por lo menos dos directorio, el directorio padre \lstinline|..| y el directorio actual \lstinline|.| como se ve a continuación.
	\begin{verbatim}
    [user@host somedir]$ ls -a
    . ..
    \end{verbatim}

	\begin{table}[ht!]
		\caption{Comandos básicos}
		\centering\footnotesize
		\begin{tabular}{cc}
			\toprule
			Comando                 & Utilidad
			\tabularnewline
			\midrule
			\lstinline|pwd|  & Muestra la ruta absoluta del directorio actual.
			\tabularnewline
			\lstinline|cd|  & Navega al home del usuario.
			\tabularnewline
			\lstinline|ls|  & Lista los archivos del directorio en el que se encuentra.
			\tabularnewline
			\lstinline|cp origen destino|  & Copia un archivo origen al destino.
			\tabularnewline
			\lstinline|mkdir nombre|  & Crea una carpeta vacía llamada nombre.
			\tabularnewline
			\lstinline|find /foo/bar -name '*.cc'|  & Busca archivos con extensión \lstinline|.cc| en la ruta \lstinline|/foo/bar|.
			\tabularnewline
			\lstinline|chmod +x ejemplo| & Dar permisos de ejecución al binario ejemplo.
			\tabularnewline
			\lstinline|rm archivo| & Elimina el inodo del archivo.
			\tabularnewline
			\lstinline|cat archivo| & Imprime el contenido de un archivo a la salida estándar.
			\tabularnewline
			\bottomrule
		\end{tabular}
	\end{table}

\end{frame}

%	\frametitle{Usando el emulador de terminal}
	% Una consola es un display del sistema
	% En todo lo que sigue, usaremos el shell \lstinline{bash}.

	% GNU General Public License v2.0
	% GNU Lesser General Public License v2.1
	% BSD 3-Clause "New" or "Revised" License

	% El terminal te da acceso al shell, que nos da acceso al sistema operativo subyacente.
	% El shell, sabe interactuar entre el usuario y el sistema operativo.

	% CLI Command Line Interface

	% pwd
	% El cursor parpadeante al final de la línea es donde se ingresan los comandos basados ​​en texto. Puede comenzar a experimentar con un comando simple, Imprimir directorio de trabajo (pwd), que muestra la carpeta actual en la que se encuentra en la pantalla. Escriba pwd y presione Entrar.

	% Todos los comandos que ingresa funcionan de la misma manera: ingresa el comando, incluye cualquier parámetro para extender el uso del comando y presiona Enter para ejecutar la línea de comando que ingresó. Escribe en la Terminal: uname -a y presiona Enter. Esto muestra información del sistema con respecto a Mint

\begin{frame}[fragile]
	\begin{lstlisting}
    gitpod ~/dune-basics $ cd
  \end{lstlisting}
\end{frame}

%\lstinputpath{../../../sandbox}

\begin{frame}[fragile]
	\frametitle{Classes}

	\lstinputlisting[
		caption={Programa \texttt{hello-linux.cc}.},
		label=hello-linux.cc,
	]{../../../sandbox/hello-linux.cc}


\end{frame}

\begin{frame}[fragile]
	\frametitle{Classes}

	\lstinputlisting[
		caption={Programa \texttt{dune-basics.cc}.},
		label=dune-basics.cc,
	]{../../../src/dune-basics.cc}

\end{frame}

\begin{frame}[fragile]
	\frametitle{Classes}

	\lstinputlisting[
		caption={Programa \texttt{dune-basics.cc}.},
		label=dune-basics.cc,
	]{../../../sandbox/dune-math-constants.cc}

\end{frame}
%https://archlinux.org/logos/archlinux-icon-crystal-256.svg

\begin{frame}
	\frametitle{El comando \lstinline{duneproject}}
	% https://gitlab.dune-project.org/core/dune-common/-/raw/master/bin/duneproject
	Es un asistente en el lenguaje \lstinline{bash} que se encuentra en \lstinline{/usr/bin/duneproject}\ldots

\end{frame}

\begin{frame}\transblindsvertical
	\frametitle{Referencias}
	%------------------------------------------------------------ 1
	\only<1>{
		\begin{itemize}
			\item Libros
			      \nocite{*}
			      \printbibliography[heading=none,keyword=book]
		\end{itemize}
	}
	%------------------------------------------------------------ 2
	\only<2>{
		\begin{itemize}
			\item Artículos
			      \printbibliography[heading=none,keyword=paper]
		\end{itemize}
	}
	%------------------------------------------------------------ 3
	\only<3>{
		\begin{itemize}
			\item Sitios web
			      \printbibliography[heading=none,keyword=online]
		\end{itemize}
	}
\end{frame}

\end{document}
%https://github.com/easybuilders/easybuild/wiki/EasyBuild-Tech-Talks-I%3A-Open-MPI
% https://www.youtube.com/playlist?list=PLagFkXs2BczZO4B_I6YhuS4xOxlSPgZuT
% https://www.youtube.com/playlist?list=PLagFkXs2BczZGsvnN7UAqd5LWfgOrat0v
% https://www.youtube.com/playlist?list=PLZRRlbOTxTmAMASXs7mAFnKkFZ3ohefTF
% https://www.youtube.com/playlist?list=PLZRRlbOTxTmBERH5Ov1kWhPlmOQAP04zi
% https://www.youtube.com/playlist?list=PLZRRlbOTxTmCaBYeLHQyqMFyI6caKsVDd
% https://www.youtube.com/playlist?list=PLZRRlbOTxTmB_gV8rLQ6N9KDX5po5eTay
% https://www.youtube.com/playlist?list=PLZRRlbOTxTmCyLkmjWg0e8cNk6rONT-mk
%https://www.historyofinformation.com/detail.php?id=872
%http://ibgwww.colorado.edu/~lessem/psyc5112/usail/concepts/hx-of-unix/unixhx.html#:~:text=One%20of%20Bell%20Laboratories%20people,version%20of%20Unix%2C%20called%20UNICS.
%https://www.britannica.com/technology/UNIX
%https://www.historyofinformation.com/detail.php?id=872
%https://homepage.cs.uri.edu/~thenry/resources/unix_art/ch02s01.html
%https://unix.org/what_is_unix/history_timeline.html